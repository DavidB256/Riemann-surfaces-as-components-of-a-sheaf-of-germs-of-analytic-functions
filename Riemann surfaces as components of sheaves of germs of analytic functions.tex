\documentclass{article}
\usepackage{amssymb}
\usepackage{amsmath}
\usepackage{setspace}
\usepackage{graphicx}
\usepackage{wasysym}
\usepackage{amsthm}
\usepackage[margin=1.5in]{geometry}

\begin{document}

\newcommand{\n}{\newline\newline}
\newcommand{\inv}{^{-1}}
\newcommand{\g}[1]{\left\langle #1\right\rangle}
\newcommand{\pd}[2]{\frac{\partial #1}{\partial #2}}
\newcommand{\mb}{\begin{pmatrix}}
\newcommand{\me}{\end{pmatrix}}
\newcommand{\lap}{\mathcal{L}}
\newcommand{\Int}{\operatorname{Int}}
\newcommand{\im}{\operatorname{im}}
\newcommand{\rk}{\operatorname{rank}}
\newcommand{\id}{\operatorname{id}}
\newcommand{\ol}{\overline}
\newcommand{\czo}{C([0, 1])}
\newcommand{\rtwo}{\mathbb{R}^2}
\newcommand{\nd}{^{(n)}}
\newcommand{\lsn}{\lim_{n\rightarrow\infty}}
\newcommand{\lsk}{\lim_{k\rightarrow\infty}}
\newcommand{\lsm}{\lim_{m\rightarrow\infty}}
\newcommand{\rbs}{\ell^\infty(\mathbb{N})}
\newcommand{\cui}{C([0, 1])}
\newcommand{\iui}{\int_0^1}
\newcommand{\ra}{\rightarrow}
\newcommand{\rs}{\mathbb{R}}
\newcommand{\ns}{\mathbb{N}}
\newcommand{\cs}{\mathbb{C}}
\newcommand{\qs}{\mathbb{Q}}
\newcommand{\fs}{\mathbb{F}}
\newcommand{\zs}{\mathbb{Z}}
\newcommand{\tl}{\triangleleft}

\theoremstyle{definition}

\newtheorem{definition}{Definition}[section]
\newtheorem{lemma}[definition]{Lemma}
\newtheorem{proposition}[definition]{Proposition}

\noindent
David Bass

\begin{center}
	Riemann Surfaces as components of sheaves of germs of analytic functions
\end{center}

\section*{Abstract}
Following the path of John B. Conway's \textit{Functions of One Complex Variable}, 2nd ed. (Springer 1978), we construct an unconventional definition of a Riemann surface as a component of a sheaf of germs of analytic functions on an open subset of the complex plane. We compare this definition to the conventional definition of a Riemann surface as a connected complex manifold of complex dimension one.

\section{Introduction}
\indent\indent 
To quote Henri Poincaré, ``Mathematics is the art of giving the same name to different things."

We assume basic knowledge of calculus, analysis, the complex numbers, and topology.

We will use ``iff" as an abbreviation for ``if and only if." We will use the symbol ``$\equiv$" to denote that two functions are equal on a given domain, e.g. ``$f\equiv g$ on $G$" signifies that $f(z) = g(z)$, $\forall z\in G$. We will use the symbols ``$\subset$" and ``$\supset$" to denote proper subsets and supersets, respectively, and will use the  symbols ``$\subseteq$" and ``$\supseteq$" to denote subsets and supersets that may be improper.

\section{Proof}
We assume basic knowledge of calculus, analysis, the complex numbers, and topology.\\
Define analyticity\\
Define sheaf\\
Explain analytic continuation\\
Monodromy theorem\\
9.5.10\\
9.5.11\\
9.5.14\\
Compare to alternate definition of Riemann surfaces\\

\begin{definition}
	Let $G$ and $H$ be subsets of $\cs$ such that $H$ is open and $H\subseteq G$. We say that $f:G\ra\cs$ is \textbf{analytic} on $H$ iff $f$ is continuously differentiable on $H$.
\end{definition}

Note that we can only define a function to be analytic on an open set. While it makes sense to claim that a function is differentiable at a point, that function must be differentiable in some neighborhood of that point in order to be analytic at that point. The impossibility of considering the analyticity of a function on a singleton set will motivate our use of germs later on.

Many sources use ``holomorphic" instead of ``analytic" in order to avoid confusion, as ``analytic" can refer to any function equal to a power series, while ``holomorphic" also specifies that the function maps between subsets of $\cs$. We will not get into the miraculous fact that any function that is analytic, as defined here, is also analytic in the broader sense of being equal to a power series.

\begin{definition}
	We define a \textbf{region} as an open connected subset of $\cs$.
\end{definition}

\begin{definition}
	We define a \textbf{component} of a set $X$ as a maximal connected subset of $X$. 
\end{definition}

E.g., $A\subseteq X$ is a component of $X$ iff $A$ is connected and there does not exist a connected subset of $X$ that contains $A$ as a proper subset. The sole component of a connected set is the set itself.

\begin{definition}
	We define a \textbf{function element} to be an ordered pair $(f, G)$, where $f$ is a function that is analytic on the region $G$. 
\end{definition}

\begin{definition}
	Let $(f, G)$ be a function element. Choose $a\in G$. We define the \textbf{germ of} $\boldsymbol f$ \textbf{at} $\boldsymbol a$, denoted $[f]_a$, to be the collection of all function elements $(g, D)$, where $a\in D$ and $f\equiv g$ in some neighborhood of $a$.
\end{definition}

\begin{lemma}
	For function elements $(f, G)$ and $(g, D)$ and $a\in G\cap D$, $(g, D)\in [f]_a$ iff $(f, G)\in [g]_a$.
\end{lemma}

\begin{proof}
	To show the forward direction of implication, suppose that $(g, D)\in [f]_a$. Then $g\equiv f$ on some neighborhood of $H$ of $a$. Reflexively, $f\equiv g$ on $H$, so $(f, G)\in [g]_a$. The backward direction of implication is shown analogously.
\end{proof}

Note that there is no notion of equality between germs at different points, e.g. we cannot say that $[f]_a = [f]_b$, for function element $(f, G)$ and distinct $a, b\in G$. However, we can intuitively define equality between germs at the same point as occurring iff the two germs contain the same collections of function elements.

\begin{definition}
	Let $\gamma:[a, b]\ra\cs$ be a continuous function, where $[a, b]\subset\rs$. We define the \textbf{path} $\boldsymbol{\gamma:[a, b]\ra\cs}$ to be the image $\gamma([a, b])$.
\end{definition}

We will only be concerned with the path as a subset of $\cs$, so this conflation of the function $\gamma$ and the path $\gamma$ will not cause any issues.

\begin{definition}
	Let $\gamma:[0, 1]\ra\cs$ be a path. Suppose that, $\forall t\in[0, 1]$, there exists a function element $(f_t, D_t)$ for which $\gamma(t)\in D_t$. Suppose that, whenever $|s-t|<\delta$, $\gamma(s)\in D_t$ and $[f_s]_{\gamma(s)} = [f_t]_{\gamma(s)}$, for some $\delta > 0$. We define the \textbf{analytic continuation of} $\boldsymbol{(f_0, D_0)}$ \textbf{along} $\boldsymbol\gamma$ to be $(f_1, D_1)$.
\end{definition}

In our discussion of analytic continuation, we will always suppose that there exists an analytic continuation for the given function element and the given path. To quote Conway, ``Since no degree of generality can be achieved which justifies the effort, no existence theorems for analytic continuations will be proved."

\begin{definition}
	Let $\gamma:[0, 1]\ra\cs$ be a path from $a$ to $b$. Let $\{(f_t, D_t) : t\in[0, 1]\}$ be an analytic continuation. We define the \textbf{analytic continuation of} $\boldsymbol{[f_0]_a}$ \textbf{along} $\boldsymbol\gamma$ to be $[f_1]_b$.
\end{definition}

In order to make this definition unambiguous, we will need to show that it does not depend on the choice of analytic continuation.

\begin{lemma}
	content...
\end{lemma}

\begin{proposition}
	Let $\gamma:[0, 1]\ra\cs$ be a path from $a$ to $b$. Let $\{(f_t, D_t): t\in[0, 1]\}$ and $\{(g_t, B_t): t\in[0, 1]\}$ be analytic continuations of $[f_0]_a=[g_0]_a$ along $\gamma$. In particular, $[f_1]_b = [g_1]_b$, i.e. two analytic continuations along the same path that start at the same germ must end at the same germ.
\end{proposition}

\begin{proof}
	Let $T = \{t\in[0, 1]:[f_t]_{\gamma(t)} = [g_t]_{\gamma(t)}$. $[f_t]_{\gamma(0)} = [f_0]_a = [g_0]_a = [g_t]_{\gamma(t)}$, so $0\in T$. In particular, $T$ is nonempty, so we can show that $T = [0, 1]$ by showing that $T$ is both open and closed in $[0, 1]$.\\\\
	To show that $T$ is open in $[0, 1]$, choose $t\in T$. By definition of analytic continuation, $\exists\delta>0$ such that $\gamma(s)\in D_t\cap B_t$, $[f_s]_{\gamma(s)} = [f_t]_{\gamma_s}$, and $[f_s]_{\gamma(s)} = [f_t]_{\gamma_s}$ whenever $|s-t|<\delta$. Let $H$ be a connected subset of $D_t\cap B_t$ containing $\gamma(s)$ and $\gamma(t)$. Since $t\in T$, $[f_t]_{\gamma(t)} = [g_t]_{\gamma(t)}$, so $f_t\equiv g_t$ on $H$. It follows that 
\end{proof}

\begin{definition}
	\normalfont
	Let $(f, G)$ be a function element. We define the \textbf{complete analytic function obtained from} $\boldsymbol{(f, G)}$ to be the collection $F$ of all germs $[g]_b$ for which there exists $a\in G$ and a path $\gamma:[0, 1]\ra\cs$ from $a$ to $b$ such that $[g]_b$ is the analytic continuation of $[f]_a$ along $\gamma$.
\end{definition}
\begin{definition}
	\normalfont
	Base space of a complete analytic function
\end{definition}
\begin{definition}
	\normalfont
	Let $F$ be a complete analytic function with base space $G$. Let $R = \{(z, [g]_z) : [g]_z\in F\}$. Let $S(G) = \{(z, [f]_z) : z\in\cs\text{, }f\text{ is analytic at }z\}$. Let $\rho:S(G)\ra\cs$ be the projection map defined by $(z, [f]_z)\mapsto z$. We define the \textbf{Riemann surface of} $\boldsymbol F$ to be the pair $(R, \rho)$.
\end{definition}

\section*{Conclusion}
Conclusion\\
Mention applications of Riemann surfaces

\section*{Works cited}
Conway\\
Second sentence of abstract

\end{document}




